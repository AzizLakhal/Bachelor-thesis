\begin{example}
	\label{simple viscosity}
	Recall the control problem from section \ref{non-smooth} and its associated value function
	\begin{equation*}
		v : \left[0, T\right] \times \mathbb{R} \to \mathbb{R}, \ (t_0, x_0) \mapsto \max(0, \normof{x_0} - (T - t_0)) \ .
	\end{equation*}
	The value function $ v $ is known to solve the Hamilton-Jacobi-Bellman type PDE \eqref{example PDE} of the form
	\begin{equation*}
		-u_t + H(t, x, -D_x u) = 0 \ ,
	\end{equation*}
	almost everywhere, with
	\begin{equation*}
		H : \left[0, T\right] \times \mathbb{R} \times \mathbb{R} \to \mathbb{R}, \ (t, x, p) \mapsto \sup\limits_{a \in \closedInterval{-1}{1}} a \cdot p = \normof{p} \ ,
	\end{equation*}
	being the corresponding Hamiltonian of our control problem.
	We shall see $ v $ is even a viscosity solution.
	
	We follow the proof of theorem \ref{classical solution HJB}, and proceed in two steps, showing $ v $ is a viscosity sub- and supersolution of \eqref{example PDE}, respectively.
	
	We first prove $ v $ is a viscosity subsolution. To this end, consider $ (t_0, x_0) \in \openInterval{0}{T} \times \mathbb{R} $ and a smooth test function $ \varphi $, s.t. $ v - \varphi $ attains a local maximum in $ (t_0, x_0) $. Having zero running costs, we obtain from the DPP \eqref{dpp_general}
	\begin{equation*}
		v(t_0, x_0) \leq v(t_0 + h, x_{\overline{a}}(t_0 + h)) \ ,
	\end{equation*}
	for any $ h \geq 0 $ and any constant control $ \overline{a} \equiv a $. As $ (t_0, x_0) $ is a local maximum of $ v - \varphi $, we have
	\begin{equation*}
		\varphi(t, x) - \varphi(t_0, x_0) \geq v(t, x) - v(t_0, x_0) \ ,
	\end{equation*}
	for points $ (t, x) $ sufficiently close to $ (t_0, x_0) $. Since the trajectory $ \simpleTrajectory{x_0}{\overline{a}} $ is continuous, we have
	\begin{equation*}
		\frac{\varphi(t_0 + h, \simpleTrajectory{x_0}{\overline{a}}(t_0 + h)) - \varphi(t_0, x_0)}{h} \geq \frac{v(t_0 + h, \simpleTrajectory{x_0}{\overline{a}}(t_0 + h)) - v(t_0, x_0)}{h} \ ,
	\end{equation*}
	for sufficiently small $ h > 0 $. It follows that
	\begin{equation*}
		\frac{\varphi(t_0 + h, \simpleTrajectory{x_0}{\overline{a}}(t_0 + h)) - \varphi(t_0, x_0)}{h} \geq 0 
	\end{equation*}
	for such $ h $. The chain rule justifies a passage to the limit $ h \searrow 0 $, which yields
	\begin{equation*}
	\varphi_t(t_0, x_0) + a D_x \varphi(t_0, x_0) \geq 0 \ ,
	\end{equation*}
	considering the dynamics of $ \simpleTrajectory{x_0}{\overline{a}} $. As the constant control was arbitrary, we conclude that
	\begin{equation*}
		- \varphi_t(t_0, x_0) + \normof{D_x \varphi(t_0, x_0)} \leq 0 \ ,
	\end{equation*}
	and that $ v $ is a viscosity subsolution of \eqref{example PDE}.
	
	We next show $ v $ is viscosity supersolution of \eqref{example PDE}.
	
	To this end, consider $ (t_0, x_0) \in \openInterval{0}{T} \times \mathbb{R} $ and a smooth test function $ \varphi $, s.t. $ v - \varphi $ attains a local minimum in $ (t_0, x_0) $. Fix $ \varepsilon > 0 $. According to the DPP \eqref{dpp_general}, there exists for every $ h > 0 $, a control $ \alpha = \alpha(h) $, such that
	\begin{equation*}
		v(t_0, x_0) > v(t_0 + h, \simpleTrajectory{x_0}{\alpha}) - \varepsilon h \ .
	\end{equation*}
	Since $ (t_0, x_0) $ is a local minimum of $ v - \varphi $,
	\begin{equation*}
		v(t_0 + h, \simpleTrajectory{x_0}{\alpha}(t_0  + h)) - v(t_0, x_0) \geq \varphi(t_0 + h, \simpleTrajectory{x_0}{\alpha}(t_0  + h)) - \varphi(t_0, x_0) \ ,
	\end{equation*}
	whenever $ \left(t_0 + h, \simpleTrajectory{x_0}{\alpha}(t_0  + h)\right) $ is sufficiently close to $ (t_0, x_0) $. As $ \alpha $ depends on $ h $, we convince ourselves the trajectories $ \simpleTrajectory{x_0}{\alpha} = \simpleTrajectory{x_0}{\alpha(h)} $ are equicontinuous in $ t_0 $. Every trajectory $ \simpleTrajectory{x_0}{\alpha} $ is described by the dynamics $ \alpha $. Therefore
	\begin{equation*}
		\normof{\simpleTrajectory{x_0}{\alpha}(t) - \simpleTrajectory{x_0}{\alpha}(t_0)} \leq \int\limits_{t_0}^{t} \normof{\alpha(t)} dt \leq (t - t_0) \ ,
	\end{equation*}
	which implies the trajectories are equicontinuous in $ t_0 $. It follows that
	\begin{equation*}
		\frac{\varphi(t_0 + h, \simpleTrajectory{x_0}{\alpha}(t_0 + h)) - \varphi(t_0, x_0)}{h} < \varepsilon \ ,
	\end{equation*}
	i.e.
	\begin{equation*}
		\frac{\varphi(t_0, x_0) - \varphi(t_0 + h, \simpleTrajectory{x_0}{\alpha}(t_0 + h))}{h} > -\varepsilon
	\end{equation*}
	for sufficiently small $ h $. Define $ \varphi_{x_0}^{\alpha} : t \mapsto \varphi(t, \simpleTrajectory{x_0}{\alpha}(t)) $ with derivative
	\begin{equation*}
		\dot{\varphi}_{x_0}^{\alpha} : t \mapsto \varphi_t(t, \simpleTrajectory{x_0}{\alpha}(t)) + H^{\alpha(t)}(t, \simpleTrajectory{x_0}{\alpha}(t), D_x \varphi(t, \simpleTrajectory{x_0}{\alpha}(t)) \ ,
	\end{equation*}
	where
	\begin{equation*}
		H^a (t, x, p) \coloneqq a \cdot p \ .
	\end{equation*}
	The fundamental theorem of calculus now gives
	\begin{equation*}
		\frac{1}{h} \int\limits_{t_0}^{t_0 + h} -\varphi_t(t, \simpleTrajectory{x_0}{\alpha}(t)) + H^{\alpha(t)}(t, \simpleTrajectory{x_0}{\alpha}(t), -D_x \varphi(t, \simpleTrajectory{x_0}{\alpha}(t)) dt > - \varepsilon \ .
	\end{equation*}
	We can rewrite $ H $ as the pointwise supremum over all $ \{H^{a}\}_{a \in \closedInterval{-1}{1}} $, and therefore
	\begin{equation*}
		\frac{1}{h} \int\limits_{t_0}^{t_0 + h} -\varphi_t(t, \simpleTrajectory{x_0}{\alpha}(t)) + H(t, \simpleTrajectory{x_0}{\alpha}(t), -D_x \varphi(t, \simpleTrajectory{x_0}{\alpha}(t)) dt > - \varepsilon \ .
	\end{equation*}
	By the mean value theorem, there is a $ \xi = \xi(h) \in \openInterval{t_0}{t_0} $, such that
	\begin{equation*}
		-\varphi_t(\xi, \simpleTrajectory{x_0}{\alpha}(\xi)) + H(\xi, \simpleTrajectory{x_0}{\alpha}(\xi), -D_x \varphi(\xi, \simpleTrajectory{x_0}{\alpha}(\xi)) dt > - \varepsilon \ .
	\end{equation*}
	Since $ \varphi $ is smooth and the trajectories $ \simpleTrajectory{x_0}{\alpha} $ are equicontinuous in $ t_0 $, a passage to the limit $ h \searrow 0 $ implies that
	\begin{equation*}
		- \varphi_t (t_0, x_0) + H(t_0, x_0, - D_x \varphi(t_0, x_0)) > - \varepsilon \ .
	\end{equation*}
	As $ \varepsilon $ was arbitrary, $ v $ is indeed a supersolution of \eqref{example PDE}, and the proof is complete.
	
	On the other hand, the second function $ f : (t_0, x_0) \mapsto \normof{x_0} - (T - t_0) $, which solves \eqref{example PDE} almost everywhere, is not a viscosity solution of \eqref{example PDE}. More precisely, we shall see that $ f $ is not a viscosity supersolution of \eqref{example PDE}.
	
	Since $ f $ is convex, its subdifferentials coincide with its subdifferentials in the sense of convex analysis \cite[see exercise 1.3, p~32]{bardi2008optimal}. Hence
	\begin{equation*}
	D^{-}f(t_0, 0) = \{1\} \times \overline{B}_{1}(0) \ ,
	\end{equation*}
	for any $ t_0 \in \openInterval{0}{T} $. Clearly, some subdifferentials of $ f $ in $ (t_0, 0) $ do not satisfy the respective supersolution condition
	\begin{equation*}
	-u_t + \normof{D_x u} \geq 0 \ .
	\end{equation*}
\end{example}