\section{Lipschitz-continuity of the value function}
\label{value lipschitz}

Solely relying on assumptions \eqref{Lipschitz} and \eqref{bounded}, we argue that the value function $ v $ is locally Lipschitz-continuous. More precisely, we derive inequality (2.36) from Theorem 2.5 in \cite[p.~165]{zhou}, namely

\begin{equation}
	\label{loc lipschitz}
	\normof{v(t_0, x_0)- v(\xi, \eta)} \leq K \left[\normof{\eta - x_0} + (1 + \max(\normof{x_0}, \normof{\eta})) \cdot \normof{\xi - t_0} \right] \ ,
\end{equation}
for some constant $ K > 0 $. 

First, we provide an upper bound for the difference of two trajectories, which on the one hand have possibly different starting conditions, but on the other, are induced by some common control $ \alpha $. That is, for two trajectories $ \trajectory{t_0}{x_0} $, $ \trajectory{\xi}{\eta} $ with respective starting conditions $ \trajectory{t_0}{x_0}(t_0) = x_0 $, $ \trajectory{\xi}{\eta}(\xi) = \eta $, we have
\begin{equation}
	\label{general gronwall}
	\normof{\trajectory{t_0}{x_0}(t) - \trajectory{\xi}{\eta}(t)} \leq K^{\prime} \left[\normof{\eta - x_0} + (1 + \max(\normof{x_0}, \normof{\eta})) \cdot \normof{\xi - t_0} \right] \ ,
\end{equation}
for some constant $ K^{\prime} > 0 $, and all $ t \in \left[0, T\right] $.

Recall the trajectories are both described by the dynamics $ b(\cdot, \cdot, \alpha(\cdot)) $, and consequently
\begin{align*}
	\trajectory{t_0}{x_0}(t) &= x_0 + \dynint{t_0}{t}{\trajectory{t_0}{x_0}} \\
	\trajectory{\xi}{\eta}(t) &= \eta + \dynint{\xi}{t}{\trajectory{\xi}{\eta}} \ .
\end{align*}

Suppose w.l.o.g. $ t_0 \leq \xi $, otherwise interchange $ (t_0, x_0) $ with $ (\xi, \eta) $. Split the integral describing $ \trajectory{t_0}{x_0}(t) $ w.r.t. to the time intervals $ \left[t_0, \xi \right] $ and $ \left[\xi, t\right] $. When doing so, the triangle inequality gives

\begin{equation*}
	\begin{split}
	\normof{\trajectory{t_0}{x_0}(t) - \trajectory{\xi}{\eta}(t)} &\leq \normof{x_0 - \eta } \\
	&+ \normedint{\xi}{t}{b(s, \trajectory{t_0}{x_0}(s), \alpha(s)) - b(s, \trajectory{\xi}{\eta}(s), \alpha(s))} \\
	&+ \normof{\dynint{t_0}{\xi}{\trajectory{t_0}{x_0}}}\ .
	\end{split}
\end{equation*}

Owing to \eqref{Lipschitz}, the second summand can be bounded from above by
\begin{equation*}
	\int\limits_{\xi}^{t} L \normof{\trajectory{t_0}{x_0}(s) - \trajectory{\xi}{\eta}(s)} \ ds \ .
\end{equation*}

The third summand rewrites to

\begin{equation*}
	\normof{\dynint{t_0}{\xi}{\trajectory{t_0}{x_0}}} = \normof{\trajectory{t_0}{x_0}(\xi) - \trajectory{t_0}{x_0}(t_0)} \ .
\end{equation*}

On account of \eqref{trajectory equicontinuous}, the third summand can be therefore bounded from above by

\begin{equation*}
	L \exp(LT) \left(1 + \normof{x_0} \right) (\xi - t_0) \ .
\end{equation*}

We finally get

\begin{equation*}
	\begin{split}
	\normof{\trajectory{t_0}{x_0}(t) - \trajectory{\xi}{\eta}(t)} &\leq \normof{x_0 - \eta } \\
	&+ L \exp(LT) \left(1 + \normof{x_0} \right) (\xi - t_0) \\
	&+ 	\int\limits_{\xi}^{t} L \normof{\trajectory{t_0}{x_0}(s) - \trajectory{\xi}{\eta}(s)} \ ds \ ,
	\end{split}
\end{equation*}

and the Gronwall-type-inequality (2.36), found in \cite[p.~42]{teschl2012ordinary}, implies

\begin{multline*}
	\normof{\trajectory{t_0}{x_0}(t) - \trajectory{\xi}{\eta}(t)} \leq \left[ \normof{\eta - x_0} + L \exp(LT) \left(1 + \normof{x_0} \right) (\xi - t_0) \right] \exp(LT) \ .
\end{multline*}

Set $ K^{\prime} \coloneqq \max \left(\exp(LT), L \exp^2(LT) \right) $ to obtain \eqref{general gronwall}, the desired inequality.

We now derive the estimate \eqref{loc lipschitz}. With this end in view, we compare the total costs associated with the trajectories $ \trajectory{t_0}{x_0} $ and $ \trajectory{\xi}{\eta} $, in other words, we consider

\begin{equation*}
	\normof{J(t_0, x_0, \alpha) - J(\xi, \eta, \alpha)} \ .
\end{equation*}
Because
\begin{align*}
	\normof{J(t_0, x_0, \alpha) - J(\xi, \eta, \alpha)} \leq &\normof{J(t_0, x_0, \alpha) - J(t_0, \eta, \alpha)} \\ 
	+ &\normof{J(t_0, \eta, \alpha) - J(\xi, \eta, \alpha)} \ ,
\end{align*}
we may study the variations with respect to different initial states and the variations with respect to different starting times separately.

We have
\begin{align*}
	\normof{J(t_0, x_0, \alpha) - J(t_0, \eta, \alpha)} \leq &\int\limits_{t_0}^{T} \normof{f(t, \trajectory{t_0}{x_0}(t), \alpha(t)) - f(t, \trajectory{t_0}{\eta}(t), \alpha(t))} dt \\
	&+ \normof{h(\trajectory{t_0}{x_0}(T) - h(\trajectory{t_0}{\eta}(T))}
	\ ,
\end{align*}
and assumption \eqref{Lipschitz}, imposed upon $ f $ and $ h $, gives
\begin{equation*}
	 \int\limits_{t_0}^{T} \normof{f(t, \trajectory{t_0}{x_0}(t), \alpha(t)) - f(t, \trajectory{t_0}{\eta}(t), \alpha(t))} \leq \int\limits_{t_0}^{T} L \normof{\trajectory{t_0}{x_0}(t) - \trajectory{t_0}{\eta}(t)}\ ,
\end{equation*}
and
\begin{equation*}
	\normof{h(\trajectory{t_0}{x_0}(T) - h(\trajectory{t_0}{\xi}(T))} \leq L \normof{\trajectory{t_0}{x_0}(T) - \trajectory{t_0}{\xi}(T)} \ ,
\end{equation*}
respectively.
Using \eqref{general gronwall}, we obtain the upper estimates
\begin{equation*}
	\int\limits_{t_0}^{T} \normof{f(t, \trajectory{t_0}{x_0}(t), \alpha(t)) - f(t, \trajectory{t_0}{\eta}(t), \alpha(t))} \leq K^{\prime\prime} \normof{\eta - x_0} \ ,
\end{equation*}
and 
\begin{equation*}
	\normof{h(\trajectory{t_0}{x_0}(T) - h(\trajectory{t_0}{\xi}(T))} \leq K^{\prime} \normof{\eta - t_0} \ ,
\end{equation*}
where $ K^{\prime\prime} \coloneqq LTK^{\prime} $. We set $ K_x \coloneqq \max(K^{\prime\prime}, K^{\prime}) $ and conclude
\begin{equation*}
	\normof{J(t_0, x_0, \alpha) - J(t_0, \eta, \alpha)} \leq K_{x} \normof{\eta - x_0} \ .
\end{equation*}
Similarly, we derive the estimate
\begin{equation*}
	\normof{J(t_0, \eta, \alpha) - J(\xi, \eta, \alpha)} \leq K_t \left(1 + \max(\normof{x_0}, \normof{\eta})) \right) \cdot \normof{\xi - t_0} \ ,
\end{equation*}
for some constant $ K_t \geq 0 $.

Altogether,
\begin{equation*}
	\normof{J(t_0, x_0, \alpha) - J(\xi, \eta, \alpha)} \leq K \left[\normof{\eta - x_0} + (1 + \max(\normof{x_0}, \normof{\eta})) \cdot \normof{\xi - t_0} \right] \ ,
\end{equation*}
as we set $ K \coloneqq \max(K_x, K_t) $.

We justify the passage to the infimum over all admissible controls in a similar manner, as we have justified the passage to the supremum in the proof of \ref{Hamiltonian gradient}. Doing so, we immediately obtain \eqref{loc lipschitz}, which completes the proof.
