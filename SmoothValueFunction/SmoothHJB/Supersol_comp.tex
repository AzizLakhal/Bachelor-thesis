\begin{lemma}
			\label{extension lemma}
			Let $ \Omega^{\prime} \subset \mathbb{R}^{N-1} $ be an open domain. Set $ \Omega \coloneqq \left( a, b \right) \times \Omega^{\prime} $. Suppose $ w \in C^{1}\left( \Omega \right) \cap C (\overline{\Omega})$ satisfies
			
			\begin{equation*}
				F(t, x, w, Dw) \leq 0
			\end{equation*}
			for all $ (t, x) \in \Omega $, where $ F : \left[ a, b \right) \times \Omega^{\prime} \times \mathbb{R} \times \mathbb{R}^{N} \to \mathbb{R} $ is a continuous function, and the mapping
			
			\begin{equation*}
				 s \mapsto F(t, x, r, s, p_1, \ldots, p_{N-1})
			\end{equation*}
			is non-increasing for every $ (t, x, r, p_1, \ldots, p_{N-1}) $. Let $ \varphi \in C^{1} \left( \Omega \right) $, s.t. its total differential continuously extends to $ \left[a, b\right) \times \Omega^{\prime} $.
			
			If $ w - \varphi $ attains a local maximum in some point $ (\overline{t}, \overline{x}) \in \left[a, b\right) \times \Omega^{\prime} $, then
			
			\begin{equation*}
				F(\overline{t}, \overline{x}, w(\overline{t}, \overline{x}), D\varphi(\overline{t}, \overline{x}))) \leq 0.
			\end{equation*}
			
			\begin{proof}
				The proof is analogous to the one given in \cite[p.~41]{bardi2008optimal}. As discussed above, the claim is an immediate consequence of Fermat's rule, if $ (\overline{t}, \overline{x}) $ is an interior extrema.
				
				Now for the critical case, in which $ (\overline{t}, \overline{x}) = (a, \overline{x}) $ is a local maximum of $ w - \varphi $ with respect to some neighbourhood $ \left( a, a + r \right] \times B_r(\overline{x}) \subset \subset \left[ a, b \right) \times \Omega^{\prime} $. 
				
				Add the penalty term $ - \frac{1}{n(t-a)} $ to $ w - \varphi $, i.e. consider the functions
				
				\begin{equation*}
					\psi_n(t, x) = ( w(t, x) - \varphi(t, x)) - \frac{1}{n(t-a)} \ .
				\end{equation*}
				
				Since $ w - \varphi $ is bounded from above by $ w(a, \overline{x}) - \varphi(a, \overline{x}) $ when restricted to $ \left( a, a + r \right] \times B_r(\overline{x}) $,
				
				\begin{equation*}
					\lim\limits_{t \searrow a} \psi_n(t, x) = - \infty
				\end{equation*}
				
				\emph{uniformly} with respect to $ x \in B_r(\overline{x}) $, and therefore $ \psi_n $ admits a local maximizer $ (t_n, x_n) \in \left( a, a + r \right] \times B_r(\overline{x}) $.
				
				After taking a subsequence if necessary, $ (t_n, x_n)_{n \in \mathbb{N}} $ converges to some point $ (t^{*}, x^{*}) \in \left[a, a +r \right] \times \overline{B_r(\overline{x}}) $. The first goal is to show that $ (t_n, x_n)_{n \in \mathbb{N}} $ actually converges to $ (a, \overline{x}) $. To this end, assume $ (a, \overline{x}) $ was a \emph{strict} local maximum. Otherwise replace $ \varphi $ by
				
				\begin{equation*}
					(t, x) \mapsto \varphi(t, x) - \frac{\lVert (t, x) - (a, \overline{x}) \rVert^2}{2},
				\end{equation*}
				
				which leaves the total derivative of $ \varphi $ in $ (a, \overline{x}) $ unchanged. 
				
				Observe that $ (\psi_n(t_n, x_n))_{n \in \mathbb{N}} $ is a non-decreasing sequence bounded from above; consequently converging to some limit $ l $. Firstly, note 
				
				\begin{equation*}
					\psi_n(t_n, x_n) \leq (w- \varphi)(t_n, x_n) \to (w-\varphi)(t^{*}, x^{*})
				\end{equation*}, 
				
				and therefore $ l \leq (w-\varphi)(t^{*}, x^{*}) $. Secondly, consider the sequence $ (s_n, z_n) $, defined by
				
				\begin{equation*}
					(s_n, z_n) \coloneqq  \left( a + \frac{1}{\sqrt{n}}, \overline{x}_1, \ldots, \overline{x}_{N-1} \right) .
				\end{equation*}
				Using that
				\begin{equation*}
					\psi_n(t_n, x_n) \geq \psi_n(s_n, z_n) = (w - \varphi)(s_n, z_n) -\frac{\sqrt{n}}{n}
				\end{equation*}
				
				conclude that $ l \geq (w- \varphi)(a, \overline{x}) $ by taking the limit $ n \to \infty $. Altogether we obtain that $ (w - \varphi)(a, \overline{x}) \leq l \leq (w - \varphi)(t^{*}, x^{*}) $. Since $ (a, \overline{x}) $ was supposed to be a \emph{strict} maximum, we know that $ (t^{*}, x^{*}) = (a, \overline{x}) $.
				
				To complete the proof, note that $ (t_n, x_n) $ is an \emph{interior} local maximum of $ \psi_n $, and therefore the total derivative of $ w $ in $ (t_n, x_n) $ equals the total derivative of the mapping 
				
				\begin{equation*}
					(t, x) \mapsto \varphi(t, x) + \frac{1}{n(t-a)}
				\end{equation*}
				
				in $ (t_n, x_n) $, i.e.
				
				\begin{equation*}
					Dw(t_n, x_n) = \left( \varphi_t(t_n, x_n) - \frac{1}{n(t-a)^2}, D_x \varphi(t_n, x_n) \right) .
				\end{equation*}
				Since $ w $ is a subsolution, we have
				
				\begin{equation*}
					F \left(t_n, x_n, w(t_n, x_n), \varphi_t(t_n, x_n) - \frac{1}{n(t-a)^2}, D_x \varphi(t_n, x_n) \right) \leq 0 \ ,
				\end{equation*}
				and the monotonicity of
				
				\begin{equation*}
					s \mapsto F( t_n, x_n, w(t_n, x_n), s, D_x \varphi(t_n, x_n ) )
				\end{equation*}
				
				yields 
				
				\begin{equation*}
					F(t_n, x_n, w(t_n, x_n), \varphi_t(t_n, x_n), D_x \varphi(t_n, x_n) ) \leq 0 \ .
				\end{equation*}
				
				Taking the limit $ n \to \infty $  proves the claim.
			\end{proof}
		\end{lemma}