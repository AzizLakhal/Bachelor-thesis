\begin{theorem}[Local comparison of classical solutions]
			\label{smooth_loc} 
			Consider the open domain $ \Omega \coloneqq \left( 0, T \right) \times B_R(z_0) $ and let $ u $, $ v \in C^{1}(\Omega) \cap C(\overline{\Omega}) $ be respectively sub-and supersolutions of some PDE of the form
			
			\begin{equation*}
				-u_t + H(t, x, -D_x u) = 0 \ .
			\end{equation*}
			If $ H : \Omega \times \mathbb{R}^{N} \to \mathbb{R} $ is $ C $-Lipschitz continuous w.r.t the gradient variable, that is if $ H $ satisfies
			
			\begin{equation}
				\tag{H1}
				\lvert H(t, x, p) - H(t, x, q) \rvert \leq C \lvert p - q \rvert
				\label{eqn:H1}
			\end{equation}
			then $ w \coloneqq u - v $ satisfies
			
			\begin{equation}
				\label{diff-PDE}
				-w_t - C \lvert D_x w \rvert \leq 0 \ .
			\end{equation}
			If additionally the subsolution $ u $ is lesser or equal to the supersolution $ v $ on $ \{ T \} \times \overline{B_R(z_0)} $, then the same applies to the whole cone $ \mathcal{C}_{z_0, R} $.
			
			\begin{proof}
				We begin the proof by showing that the difference $ w $ satisfies \ref{diff-PDE}. Since $ u $ and $ v $ are respectively sub-and supersolutions, we have by definition
				
				\begin{equation*}
					-u_t + H(t, x, -D_x u) \leq 0 \leq -v_t + H(t, x, -D_x v) .
				\end{equation*}
				
				Reordering the Hamiltonians on the right-hand-side, and the derivatives on the left one, yields
				
				\begin{equation*}
					-w_t \leq \lvert H(t, x, -D_x u) - H(t, x, -D_x v) \rvert \ .
				\end{equation*}
				
				The Lipschitz-continuity w.r.t the gradient variable \eqref{eqn:H1} immediately gives
				
				\begin{equation*}
					-w_t \leq C \lvert D_x w \rvert \ ,
				\end{equation*}
				
				which is obviously equivalent to inequality \ref{diff-PDE}.
				
				To conclude $ u \leq v $, i.e $ w \leq 0 $ on $ \mathcal{C}_{x_0, R} $, compare $ w $ with strict supersolutions 
				
				\begin{align*}
				\phi_{\delta} : \left[ t_{\delta}, T \right] \times \overline{B_R(z_0)} &\to \mathbb{R} \\
				(t, x) &\mapsto \chi_{\delta} \left( \lvert x - x_0 \rvert + C(T - t) \right) + \delta (T - t)
				\end{align*}
				of \ref{diff-PDE}, where $ t_{\delta} \coloneqq T - \frac{R - \delta}{C} $ and $ \chi_{\delta} $ is some smooth function, with
				
				\begin{align*}
					\chi_{\delta}(y) = \begin{cases}
					0 &\hbox{, if } y \leq R - \delta \\
					\max\limits_{\left[0, T\right] \times \overline{B_z(x_0)}} w &\hbox{, if }y \geq R
					\end{cases}
				\end{align*}
				
				as described in \cite[p.~73]{barles}. An explicit computation of the difference quotients of $ \phi_{\delta} $ w.r.t the state variables, shows that $ \phi_{\delta} $ is indeed a smooth, strict supersolution of \ref{diff-PDE} in $ \left(t_{\delta}, T \right) \times B_R(z_0) $, and that $ D \phi_{\delta} $ continuously extends to $ \left[t_{\delta}, T \right) \times B_R(z_0) $. In view of the conflicting sub-and supersolution conditions of $ w $ and $ \phi_{\delta} $ in combination with lemma \ref{extension lemma}, the continuous function $ w - \phi_{\delta} $ attains its global maximum over the compact set $ \left[t_{\delta}, T \right] \times \overline{B_R(z_0)} $, in $ \left[t_{\delta}, T \right] \times \partial B_R(z_0) $. By construction of $ \phi_{\delta} $, we have $ w \leq \phi_{\delta} $, i.e $ w - \phi_{\delta} \leq 0 $ in $ \left[t_{\delta}, T \right] \times \partial B_R(z_0) $, and therefore $ w - \phi_{\delta} \leq 0 $ on the whole domain $ \left[ t_{\delta}, T \right] \times \overline{B_R(z_0)} $.
				Now observe that $ \phi_{\delta} \leq 0 $ in $ \mathcal{C}_{z_0, R - \delta} $ by construction, and since $ w \leq \phi_{\delta} $, the same applies to $ w $. Taking the limit $ \delta \searrow 0 $, completes the proof.
			\end{proof}
		\end{theorem}