\begin{theorem}[Local comparison of classical solutions]
			\label{smooth_loc} 
			Consider the open domain $ \Omega \coloneqq \left( 0, T \right) \times B_R(z_0) $ and let $ u $, $ v \in C^{1}(\Omega) \cap C(\overline{\Omega}) $ be sub-and supersolutions, respectively, to some PDE of the form
			
			\begin{equation*}
				-u_t + H(t, x, -D_x u) = 0 \ .
			\end{equation*}
			If $ H : \Omega \times \mathbb{R}^{N} \to \mathbb{R} $ is $ C $-Lipschitz continuous w.r.t. the gradient variable, that is if $ H $ satisfies
			
			\begin{equation}
				\tag{H1}
				\lvert H(t, x, p) - H(t, x, q) \rvert \leq C \lvert p - q \rvert \ ,
				\label{eqn:H1}
			\end{equation}
			then $ w \coloneqq u - v $ satisfies
			
			\begin{equation}
				\label{diff-PDE}
				-w_t - C \lvert D_x w \rvert \leq 0 \ .
			\end{equation}
			If additionally the subsolution $ u $ is lesser or equal to the supersolution $ v $ on $ \{ T \} \times \overline{B}_R (z_0) $, then the same applies to the whole region $ \mathcal{C}_{z_0, R} $.
			
			\begin{proof}
				We begin the proof by showing that the difference $ w $ satisfies \eqref{diff-PDE}. Since $ u $ and $ v $ are respectively sub-and supersolutions, we have by definition
				
				\begin{equation*}
					-u_t + H(t, x, -D_x u) \leq 0 \leq -v_t + H(t, x, -D_x v) .
				\end{equation*}
				
				Reordering the Hamiltonians on the right-hand-side, and the partial derivatives w.r.t. time on the left one, yields
				
				\begin{equation*}
					-w_t \leq \lvert H(t, x, -D_x u) - H(t, x, -D_x v) \rvert \ .
				\end{equation*}
				
				The Lipschitz-continuity w.r.t. the gradient variable \eqref{eqn:H1} immediately gives
				
				\begin{equation*}
					-w_t \leq C \lvert D_x w \rvert \ ,
				\end{equation*}
				
				which is obviously equivalent to inequality \eqref{diff-PDE}.
				
				We now conclude $ u \leq v $, i.e. $ w \leq 0 $ on $ \mathcal{C}_{z_0, R} $. Firstly, compare $ w $ with the strict supersolutions 
				
				\begin{align*}
				\varphi_{\delta} : \left[ t_{\delta}, T \right] \times \overline{B}_R (z_0) &\to \mathbb{R} \\
				(t, x) &\mapsto \chi_{\delta} \left( \lvert x - z_0 \rvert + C(T - t) \right) + \delta (T - t)
				\end{align*}
				of \eqref{diff-PDE}, where $ t_{\delta} \coloneqq T - \frac{R - \delta}{C} $ and $ \chi_{\delta} $ is some smooth, non-decreasing function, with
				
				\begin{align*}
					\chi_{\delta}(y) = \begin{cases}
					0 &\hbox{, if } y \leq R - \delta \\
					M \coloneqq \max\limits_{\left[0, T\right] \times \overline{B}_R (z_0)} w &\hbox{, if }y \geq R
					\end{cases}
				\end{align*}
				
				as described in \cite[p.~73]{barles}. We may assume w.l.o.g. that $ M \geq 0 $, otherwise there is nothing to prove. As shown in \ref{smoothness_supersolutions}, the function $ \varphi_{\delta} $ is indeed a smooth, strict supersolution of \eqref{diff-PDE} over $ \left(t_{\delta}, T \right) \times B_R(z_0) $. Furthermore, its differential $ D \varphi_{\delta} $ continuously extends to $ \left[t_{\delta}, T \right) \times B_R(z_0) $, maintaining the respective supersolution condition. We are now in a position to invoke lemma \ref{extension lemma}, excluding the possibility of $ w - \phi_{\delta} $ attaining a local maximum in $ \left[t_{\delta}, T \right) \times B_R (z_0) $. Suppose for the sake of contradiction, the function $ w - \phi_{\delta} $ attained a local maximum in $ (t^{*}, x^{*}) \in \left[t_{\delta}, T \right) \times B_R (z_0) $. This would imply by lemma \ref{extension lemma} that
				\begin{equation*}
					-(\varphi_{\delta})_t(t^*, x^*) - \lvert D_x \varphi_{\delta}(t^*, x^*) \rvert \leq 0 \ ,
				\end{equation*}
				contradicting the fact, that $ \varphi_{\delta} $ is \emph{strict} supersolution of \eqref{diff-PDE} over $ \left[t_{\delta}, T \right) \times B_R (z_0) $.
				Global maxima of $ w - \varphi_{\delta} $, in particular, can therefore only occur at either $ \{T\} \times \overline{B}_R (z_0) $ or $ \left[t_{\delta}, T\right] \times \partial B_R(z_0) $. In either situation, we now deduce that such a maximum is non-positive, which implies $ w \leq \phi_{\delta} $ over $  \left[ t_{\delta}, T \right] \times \overline{B}_R (z_0) $. If the global maximum is attained in $ \{T\} \times \overline{B}_R (z_0) $, the boundary condition gives $ w \leq 0 $, while $ \varphi_{\delta} \geq 0 $, as $ \chi_{\delta} $ is non-decreasing. The first case has therefore been dealt with. Now suppose the global maximum is attained in $ \left[t_{\delta}, T\right] \times \partial B_R(z_0) $. The function $ \varphi_{\delta} $ constantly evaluates to $ M $ over $ \left[t_{\delta}, T\right] \times \partial B_R(z_0) $, while $ w \leq M $ holds by the definition of $ M $. This immediately implies that the considered maximum is non-positive. As a global maximum has to be attained over the compact set $ \left[t_{\delta}, T\right] \times \overline{B}_R (z_0)$, we conclude $ w - \varphi_{\delta} \leq 0$, i.e. $ w \leq \varphi_{\delta} $, on the whole domain $ \left[t_{\delta}, T\right] \times \overline{B}_R (z_0) $.
				
				We finally observe that $ \varphi_{\delta}(t, x) = 0 $ holds in any point $ (t, x) \in \left[t_{\delta}, T\right] \times \overline{B}_R(z_0)$, that satisfies the inequality
				\begin{equation*}
					\lvert x - z_0 \rvert + C(T - t) \leq (R - \delta) \ .
				\end{equation*}
				Letting $ \delta $ tend to zero completes the proof.
			\end{proof}
		\end{theorem}