%%%%%%%%%%%%%%%%%%%%%%%%%%%%%%%%%%%%%%%%%%%%%%%%%%%%%%
%%  Dateiname: preprint_sample.tex
%%  Datum:     24.04.2003
%%  Autor:     H. Cr"oni, M. Bebendorf, R. Kirsch. J. Weickert
%%  E-Post:    holger@math.uni-sb.de & kirsch@num.uni-sb.de
%%
%%  Sie brauchen zusaetzlich die Datei 'titelblatt.sty'
%%  und 'eule.ps'. Dann koennen sie 'latex PPsample[1-5].tex'
%%  eintippen und die ersten Seiten eines Mustervorabdrucks
%%  sollten entstehen. 
%%%%%%%%%%%%%%%%%%%%%%%%%%%%%%%%%%%%%%%%%%%%%%%%%%%%%%


\documentclass{report}

\usepackage[T1]{fontenc}
\usepackage{enumerate}

\usepackage{epsfig}
\usepackage{amssymb}
\usepackage{amsmath}
\usepackage{amsthm}
\usepackage{mathtools}
\usepackage{dirtytalk}
\usepackage{import}
\usepackage{appendix}
\usepackage{hyperref}
\usepackage{titelblatt}
\usepackage{url}
\usepackage{ragged2e}
\usepackage{ellipsis}

\theoremstyle{definition}
\newtheorem{definition}{Definition}[section]

\theoremstyle{remark}
\newtheorem*{remark}{Remark}

\theoremstyle{plain}
\newtheorem{theorem}{Theorem}[section]
\newtheorem{proposition}{Proposition}[section]
\newtheorem{corollary}{Corollary}[theorem]
\newtheorem{lemma}[theorem]{Lemma}

\theoremstyle{definition}
\newtheorem{example}{Example}[section]

\DeclareMathOperator*{\argmax}{arg\,max}

\newcommand{\pathint}[3]{\int\limits^{{#3}}_{{#2}} f(t, x^{#1}_{x_0}(t), #1(t))dt}

\newcommand{\dynint}[3]{\int\limits_{#1}^{#2} b(s, #3(s), \alpha(s)) \ ds}

\newcommand{\normof}[1]{\left\lvert {#1} \right\rvert}

\newcommand{\normedint}[3]{\int\limits_{#1}^{#2} \normof{#3} \ ds}

\newcommand{\controlspace}[1][0]{\mathcal{V}\left[ {#1}, T \right]}

\newcommand{\trajectory}[2]{x_{#1, #2}^{\alpha}}

\newcommand{\simpleTrajectory}[2]{x_{#1}^{#2}}

\newcommand{\overlinedPair}[2]{\left(\overline{#1}, \overline{#2}\right)}

\newcommand{\closedInterval}[2]{\left[ #1, #2 \right]}

\newcommand{\openInterval}[2]{\left(#1, #2\right)}

\begin{document}


%%%%%%%%%%%%%%%%% Hier wird der Titel erstellt %%%%%%%%%%%%%%%%%%%%%%%%%%%%%%%%

\Panzahl    {1}                   %% number of authors (up to 5 authors supp.)
\Pautor     {Mohamed Aziz Lakhal}       %% name of 1st author

%% address of 1st author
\Panschrift {Saarland University \\ Department of Mathematics \\
             P.O. Box 15 11 50   \\ 66041 Saarbr\"ucken \\
             Germany}
\Pepost     {s8molakh@stud.uni-saarland.de}     %% email of 1st author 


\Ptitel     {Viscosity solutions of Hamilton-Jacobi-Bellman equations arising in the control of ordinary differential equations}  %% title


\Pjahr      {2020}               %% year of publ.
\Pnummer    {???}                 %% preprint no. (ask "Preprintbeauftragten")

\Pdatum     {\today}             %% date of submission to journal, default: today 



%%%%%%%%%%%%%%%%%%%%%%%%%%%%%%%%%%%%%%%%%%%%%%%%%%%%%%%%%%%%%%%%%%%%%%%%%

\Ptitelseite                 %% generates the title pages 

%%%%%%%%%%%%%%%%%%%%%%%%%%%%%%%%%%%%%%%%%%%%%%%%%%%%%%%%%%%%%%%%%%%%%%%%%

\tableofcontents

\import{Outline/}{outline}

\import{Acknowledgements/}{acknowledgements}

\import{ControlProblems/}{ControlProblems}

\import{SmoothValueFunction/}{SmoothValueFunction}

\import{MotivationViscosity/}{MotivationViscosity}

\import{ViscosityValueFunction/}{ViscosityValueFunction}

\import{Outlook/}{outlook}

\import{Appendix/}{appendix}

\bibliographystyle{plain}

\bibliography{./bibliography.bib}

\end{document}
