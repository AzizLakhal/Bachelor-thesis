\newpage

One of the classical approaches to control problems, namely the synthesis procedure, relies on describing the associated value function as a solution to some terminal value problem. As the value function already fails to be smooth for simple control problems, it is necessary to consider solutions in a weaker sense.
Crandall's \cite{lions}, Lions' \cite{lions1982fully, lions1982generalized} and Evans' \cite{crandall1984some} notion of viscosity solutions has proved to be particularly, but not exclusively, suitable for this purpose. Accordingly, it was awarded the Fields Medal.

This work was intended as an attempt to motivate the notion of viscosity solutions from a control-theoretical point of view, and to emphasize its properties guaranteeing the uniqueness of aforementioned terminal value problem. To accomplish this goal, we mainly relied on Barles' lecture notes \cite{barles} and the textbooks by Yong and Zhou \cite{zhou} and Bardi and Capuzzo-Dolcetta \cite{bardi2008optimal}.

In the first chapter we define the specific class of the control problems we are dealing with in this thesis, alongside their associated value functions. We will then show, that such a value function satisfies a functional equation, known as the dynamic programming principle.

Throughout the second chapter, we will assume the value function of the control problem to be smooth. Under latter assumption, we will use the dynamic programming principle, to prove that such a value function is a solution of a terminal value problem with a partial differential equation of Hamilton-Jacobi-Bellman type. We will then discuss the practical importance of guaranteeing uniqueness, which are related to the terminal value problem. Addressing the question of uniqueness, we present a classical local comparison result, and immediately extend it to a global one, closing the issue. By considering a simple instance of the control problems at hand, we realize that it is not sensible to assume the value function to be smooth. We conclude the chapter on a rather pessimistic note, as the invocation of the uniqueness result becomes invalid, when generalizing the sense of solution in the most obvious way.

We start the third chapter in a retrospective manner, analyzing the properties of the classical derivative on which rely the proofs given in the previous chapter. Based on the insights gained, we introduce the notion of viscosity solution. Chronologically, we swiftly adjust the results within the new framework, with the exception of the local comparison result. We discuss the issues encountered when dealing with the local comparison result, suggesting an alternative strategy. Finally, present the doubling of variables method, which has become the customary way to provide comparison results and tackling questions regarding the uniqueness of viscosity solutions.

In the fourth chapter, we recall the classical synthesis procedure and mention related difficulties. We leave it an open question, how the classical synthesis procedure can be adapted to merely continuous value functions. In view of the comparison results for viscosity solutions, the last chapter is topped off with a brief exposition of sufficient optimality conditions which do not require a smooth value function.





