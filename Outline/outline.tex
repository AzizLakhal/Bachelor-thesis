\chapter*{Outline}

One of the classical approaches to control problems, namely the synthesis procedure, relies on describing the associated value function as a solution to some terminal value problem. As the value function already fails to be smooth for simple control problems, it is necessary to consider solutions in a weaker sense.
Crandall's \cite{lions}, Lions' \cite{lions1982fully, lions1982generalized} and Evans' \cite{crandall1984some} notion of viscosity solutions has proved to be particularly, but not exclusively, suitable for this purpose.

Our aim is to motivate the notion of viscosity solutions from a control-theoretical point of view, emphasizing its properties guaranteeing the uniqueness of aforementioned terminal value problem. 

In the first chapter we define the specific class of control problems considered to the end of this thesis, alongside their associated value functions. We will then show, that such a value function satisfies a functional equation, known as the dynamic programming principle.

Throughout the second chapter, we will assume the value function of such a problem to be smooth. Under latter assumption, we will use the dynamic programming principle, to prove that such a value function is a solution of a terminal value problem with Hamilton-Jacobi-Bellman-type PDE. We will then stress the practical importance of uniqueness guarantees, related to the terminal value problem. Addressing the question of uniqueness, we present a classical local comparison result, and immediately extend it to a global one, closing the issue. By considering a simple instance of the control-problems at hand, we realize that it is not sensible to assume the value function to be smooth. We conclude the chapter on a rather pessimistic note, as the invocation of our uniqueness result becomes invalid, when generalizing the sense of solution in the most obvious way.

We start the third chapter in a retrospective manner, analyzing the properties of the classical derivative on which rely the proofs given in the previous chapter. Based on the insights gained, we introduce the notion of viscosity solution. Chronologically, we seamlessly adjust our results within our new framework, with the exception of our local comparison result. We discuss the issues encountered when dealing with the latter, hinting at an alternative strategy. Finally, we get acquainted with the doubling of variables method, which has become the customary way to provide comparison results and tackling questions regarding the uniqueness of viscosity solutions.

In the fourth chapter, we recall the classical synthesis procedure and mention related difficulties. We leave it an open question, how the classical synthesis procedure can be adapted to merely continuous value functions. In view of our comparison results for viscosity solutions, the last chapter is topped off with a brief exposition of sufficient optimality conditions which do not require a smooth value function.





